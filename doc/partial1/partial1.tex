
%----------------------------------------------------------------------------------------
%	PACKAGES AND DOCUMENT CONFIGURATIONS
%----------------------------------------------------------------------------------------

\documentclass[a4paper]{article} %Article class
\usepackage[utf8]{inputenc} %utf-8 Encoding
\usepackage{graphicx} % Required for the inclusion of images
\usepackage{float} %requiered for image positioning
\graphicspath{{Figures/}} % Set the default folder for images
\usepackage{amsmath} % Required for some math elements 
\usepackage[english]{babel} % Language 
\setlength\parindent{0pt} % Removes all indentation from paragraphs
\usepackage{xcolor}	
\usepackage{listings} %Requiered for code inclusion
\renewcommand{\labelenumi}{\alph{enumi}.} % Make numbering in the enumerate environment by letter rather than number (e.g. section 6)
\usepackage[
backend=biber,
style=alphabetic,
sorting=ynt
]{biblatex}
\addbibresource{sample.bib}

\usepackage{color}
\lstset{ %
	language=R,                     % the language of the code
	basicstyle=\footnotesize,       % the size of the fonts that are used for the code
	numbers=left,                   % where to put the line-numbers
	numberstyle=\tiny\color{gray},  % the style that is used for the line-numbers
	stepnumber=1,                   % the step between two line-numbers. If it's 1, each line
	% will be numbered
	numbersep=5pt,                  % how far the line-numbers are from the code
	backgroundcolor=\color{white},  % choose the background color. You must add \usepackage{color}
	showspaces=false,               % show spaces adding particular underscores
	showstringspaces=false,         % underline spaces within strings
	showtabs=false,                 % show tabs within strings adding particular underscores
	frame=single,                   % adds a frame around the code
	rulecolor=\color{black},        % if not set, the frame-color may be changed on line-breaks within not-black text (e.g. commens (green here))
	tabsize=2,                      % sets default tabsize to 2 spaces
	captionpos=b,                   % sets the caption-position to bottom
	breaklines=true,                % sets automatic line breaking
	breakatwhitespace=false,        % sets if automatic breaks should only happen at whitespace
	title=\lstname,                 % show the filename of files included with \lstinputlisting;
	% also try caption instead of title
	keywordstyle=\color{blue},      % keyword style
	commentstyle=\color{dkgreen},   % comment style
	stringstyle=\color{mauve},      % string literal style
	escapeinside={\%*}{*)},         % if you want to add a comment within your code
	morekeywords={*,...},
	% if you want to add more keywords to the set
	alsoletter={.}        % if you want to add more keywords to the set
} 

%----------------------------------------------------------------------------------------
%	DOCUMENT INFORMATION
%----------------------------------------------------------------------------------------

\title{Network Intruder Detection \\
\large APA Practical Work \\
Universitat Politècnica de Barcelona} % Title
\author{Lluc Bové \& Aleix Trasserra} % Author name

\date{Q1 2016-17}

\begin{document}

\maketitle % Insert the title, author and date


%----------------------------------------------------------------------------------------
%	SECTION 1
%----------------------------------------------------------------------------------------

\section{Introduction}
\subsection{The data}
%----------------------------------------------------------------------------------------
%	SECTION 2
%----------------------------------------------------------------------------------------

\section{Related Work}

%----------------------------------------------------------------------------------------
%	SECTION 3
%----------------------------------------------------------------------------------------

\section{Data Exploration Process}
In this section we will take a first look to our data, in order to know what are we going to analyse. First we'll preprocess it, this includes detecting outliers, missing values... And deleting them. Then we are going to perform a feature extraction and selection. This consists in deleting variables that aren't necessary and deriving new ones. Then we'll visualize our data using FDA and finally we'll do a clustering. All this process can be followed using the script provided called \textit{dataExploration.R}.\\

\subsection{Preprocessing}
The original data has two forms. The first one has a dimension of 5 million rows approximately and the other is a reduction of 10\%. We assured that this purge didn't affect the proportion of less used attack types, and it didn't, so from now on we will work with this reduction.\\

First of all when we analyse our dataset we see that variables have no name, so we name them. Then we execute \lstinline|summary| to detect abnormal things on our data, like strange means, medians, extreme values, errors... So we first realize that our categorical variables aren't factors, and they have no text. To solve this first we declare all categorical as factors and we use \lstinline|TRUE| and \lstinline|FALSE| for boolean ones, and we use names for the others. We realize that some variables do not vary so we delete them. We detected an error with a boolean factor, it had missing values so we deleted the rows.


\subsection{Feature extraction/selection}
First we need to define the target variable. We decide to use the main attack in the network. This is a category for each attack type. The variable is defined as table \ref{table:main} shows. 
\begin{table}
	\begin{tabular}{ll}
		Main Attack & Attack Type                                                            \\ \hline
		DOS         & back, land, neptune, smurf, teardrop                                       \\
		U2R         & buffer overflow, loadmodule, perl, rootkit                               \\
		R2L         & ftp\_write, guess password, imap, multihop, phf, spy, warezclient, warezmaster \\
		probe       & ipsweep, nmap, portsweep, satan                                           \\
		normal      & normal                                                                 \\
	\end{tabular}
	\caption{This table shows the main attack classification}
	\label{table:main}
	
\end{table}
Now we have this categorization so we don't need attack type any more but we save it because it may be a useful alternative target respect main attack. 


\subsection{Descriptive Analysis}
We'll proceed to do a descriptive analysis of our data. To make our graphs more beautiful we use \textit{ggplot2}\cite{ggplot}. For every numerical variable we do a histogram and a boxplot. We realize that the median of many variables is 0, that is, half of the data is zero. This is because some variables are TCP camps that hardly ever are used but when are used they are important. Another cause is that many variables depend on if the communication is between the host and the server or the server and the host. This is why we want to show this variables without including the zero's. We also applied a $log$ transformation. Now we will show some of this plots:\\

\begin{figure}[H]
	\centering
	\includegraphics[scale=0.25]{duration.png}
	\caption{This figure shows descriptive graphs for the duration variable}
	\label{fig:duration}
\end{figure}

Figure \ref{fig:duration} shows the boxplot and histogram for the variable \textit{duration}, that is the duration of the TCP connection. We can see that most TCP connections have a duration lower than zero. This causes that histogram and boxplot are difficult to visualize. If we don't display the zero's we fix this. This variable also is advantaged by a $log$ transformation. \\

Many variables have the same structure as the ones before so we don't include them in this analysis. The next thing that we'll do is the descriptive analysis plots for the qualitative variables. We analyse them using contingency tables and bar plots.

Figure \ref{fig:barplots} shows a barplot for main attack variable. We want to emphasize that it has an unequal distribution. Most connections are DOS attacks, followed by normal and then probe. R2L and U2L are in a tiny proportion.

\begin{figure}[H]
	\centering
	\includegraphics[scale=0.5]{bar_main_attack.png}
	\caption{Barplot for main attack}
	\label{fig:barplots}
\end{figure}
We made contingency tables from some of the most relevant variables, against the main attack variable. This variables are protocol, flags, service and superuser attempted. Figure \ref{fig:cont} shows the result. 

\begin{figure}[H]
	\centering
	\includegraphics[scale=0.85]{cont.png}
	\caption{Contingency tables represented }
	\label{fig:cont}
\end{figure}

We can observe that for example, DOS attacks and normal connections have similar flags, in probe attacks the flag \textit{RESET} is very often, and in \textit{U2R} it predominates \textit{REJ}. In all types of connections the protocols used are similar. The services are almost the same with normal connections, DOS and PROBE attacks. They differ with the R2L and U2L attacks and the service that predominates in U2L is \textit{eco\_i}. The only tpye of connection that does request super user is DOS attacks.
\subsection{Visualization}
To do visualization we use the Fisher Discriminant Analysis or FDA. We apply it using only numerical variables and our classes are the different values of main attack. The visualization of the first two components can be observed in figure \ref{fig:fda}.

\begin{figure}[H]
	\centering
	\includegraphics[scale=0.45]{LDA.png}
	\caption{Barplots for all categorical variables}
	\label{fig:fda}
\end{figure}

\begin{figure}[H]
	\centering
	\includegraphics[scale=0.45]{LDA_log.png}
	\caption{Barplots for all categorical variables}
	\label{fig:fda_log}
\end{figure}
The classes don't seem very separated. In fact, if we didn't use colours we wouldn't have seen the classes separated. This indicates us that the prediction task can be a big challenge. We decided to apply logarithms to all the variables and try the analysis again, the result is in \ref{fig:fda_log}. This representation is slightly better but not very good though. The trace of the two first components sum 98.49\%, and without the $log$ 98.02\%. So this projection is more accurate with 2 dimensions. 

\subsection{Clustering}







\medskip
%----------------------------------------------------------------------------------------
%	BIBLIOGRAPHY
%----------------------------------------------------------------------------------------

\printbibliography

%----------------------------------------------------------------------------------------


\end{document}