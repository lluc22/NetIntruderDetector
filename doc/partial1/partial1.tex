
%----------------------------------------------------------------------------------------
%	PACKAGES AND DOCUMENT CONFIGURATIONS
%----------------------------------------------------------------------------------------

\documentclass[a4paper]{article} %Article class
\usepackage[utf8]{inputenc} %utf-8 Encoding
\usepackage{graphicx} % Required for the inclusion of images
\graphicspath{{Figures/}} % Set the default folder for images
\usepackage{natbib} % Required to change bibliography style to APA
\usepackage{amsmath} % Required for some math elements 
\usepackage[english]{babel} % Language 
\setlength\parindent{0pt} % Removes all indentation from paragraphs
\usepackage{xcolor}	
\usepackage{listings} %Requiered for code inclusion
\renewcommand{\labelenumi}{\alph{enumi}.} % Make numbering in the enumerate environment by letter rather than number (e.g. section 6)

\usepackage{color}
\lstset{ %
	language=R,                     % the language of the code
	basicstyle=\footnotesize,       % the size of the fonts that are used for the code
	numbers=left,                   % where to put the line-numbers
	numberstyle=\tiny\color{gray},  % the style that is used for the line-numbers
	stepnumber=1,                   % the step between two line-numbers. If it's 1, each line
	% will be numbered
	numbersep=5pt,                  % how far the line-numbers are from the code
	backgroundcolor=\color{white},  % choose the background color. You must add \usepackage{color}
	showspaces=false,               % show spaces adding particular underscores
	showstringspaces=false,         % underline spaces within strings
	showtabs=false,                 % show tabs within strings adding particular underscores
	frame=single,                   % adds a frame around the code
	rulecolor=\color{black},        % if not set, the frame-color may be changed on line-breaks within not-black text (e.g. commens (green here))
	tabsize=2,                      % sets default tabsize to 2 spaces
	captionpos=b,                   % sets the caption-position to bottom
	breaklines=true,                % sets automatic line breaking
	breakatwhitespace=false,        % sets if automatic breaks should only happen at whitespace
	title=\lstname,                 % show the filename of files included with \lstinputlisting;
	% also try caption instead of title
	keywordstyle=\color{blue},      % keyword style
	commentstyle=\color{dkgreen},   % comment style
	stringstyle=\color{mauve},      % string literal style
	escapeinside={\%*}{*)},         % if you want to add a comment within your code
	morekeywords={*,...},
	% if you want to add more keywords to the set
	alsoletter={.}        % if you want to add more keywords to the set
} 

%----------------------------------------------------------------------------------------
%	DOCUMENT INFORMATION
%----------------------------------------------------------------------------------------

\title{Network Intruder Detection \\
\large APA Practical Work \\
Universitat Politècnica de Barcelona} % Title
\author{Lluc Bové \& Aleix Trasserra} % Author name

\date{Q1 2016-17}

\begin{document}

\maketitle % Insert the title, author and date


%----------------------------------------------------------------------------------------
%	SECTION 1
%----------------------------------------------------------------------------------------

\section{Introduction}
\subsection{The data}
%----------------------------------------------------------------------------------------
%	SECTION 2
%----------------------------------------------------------------------------------------

\section{Related Work}

%----------------------------------------------------------------------------------------
%	SECTION 3
%----------------------------------------------------------------------------------------

\section{Data Exploration Process}
In this section we will take a first look to our data, in order to know what are we going to analyse. First we'll preprocess it, this includes detecting outliers, missing values... And deleting them. Then we are going to perform a feature extraction and selection. This consists in deleting variables that aren't necessary and deriving new ones. Then we'll visualize our data using FDA and finally we'll do a clustering. All this process can be followed using the script provided called \textit{dataExploration.R}
\subsection{Preprocessing}
\subsection{Feature extraction/selection}
\subsection{Visualization}
\subsection{Clustering}


%----------------------------------------------------------------------------------------
%	BIBLIOGRAPHY
%----------------------------------------------------------------------------------------

\bibliographystyle{apalike}

\bibliography{sample}

%----------------------------------------------------------------------------------------


\end{document}