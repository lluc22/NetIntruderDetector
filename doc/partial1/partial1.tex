
%----------------------------------------------------------------------------------------
%	PACKAGES AND DOCUMENT CONFIGURATIONS
%----------------------------------------------------------------------------------------

\documentclass[a4paper]{article} %Article class
\usepackage[utf8]{inputenc} %utf-8 Encoding
\usepackage{graphicx} % Required for the inclusion of images
\graphicspath{{Figures/}} % Set the default folder for images
\usepackage{amsmath} % Required for some math elements 
\usepackage[english]{babel} % Language 
\setlength\parindent{0pt} % Removes all indentation from paragraphs
\usepackage{xcolor}	
\usepackage{listings} %Requiered for code inclusion
\renewcommand{\labelenumi}{\alph{enumi}.} % Make numbering in the enumerate environment by letter rather than number (e.g. section 6)
\usepackage[
backend=biber,
style=alphabetic,
sorting=ynt
]{biblatex}
\addbibresource{sample.bib}

\usepackage{color}
\lstset{ %
	language=R,                     % the language of the code
	basicstyle=\footnotesize,       % the size of the fonts that are used for the code
	numbers=left,                   % where to put the line-numbers
	numberstyle=\tiny\color{gray},  % the style that is used for the line-numbers
	stepnumber=1,                   % the step between two line-numbers. If it's 1, each line
	% will be numbered
	numbersep=5pt,                  % how far the line-numbers are from the code
	backgroundcolor=\color{white},  % choose the background color. You must add \usepackage{color}
	showspaces=false,               % show spaces adding particular underscores
	showstringspaces=false,         % underline spaces within strings
	showtabs=false,                 % show tabs within strings adding particular underscores
	frame=single,                   % adds a frame around the code
	rulecolor=\color{black},        % if not set, the frame-color may be changed on line-breaks within not-black text (e.g. commens (green here))
	tabsize=2,                      % sets default tabsize to 2 spaces
	captionpos=b,                   % sets the caption-position to bottom
	breaklines=true,                % sets automatic line breaking
	breakatwhitespace=false,        % sets if automatic breaks should only happen at whitespace
	title=\lstname,                 % show the filename of files included with \lstinputlisting;
	% also try caption instead of title
	keywordstyle=\color{blue},      % keyword style
	commentstyle=\color{dkgreen},   % comment style
	stringstyle=\color{mauve},      % string literal style
	escapeinside={\%*}{*)},         % if you want to add a comment within your code
	morekeywords={*,...},
	% if you want to add more keywords to the set
	alsoletter={.}        % if you want to add more keywords to the set
} 

%----------------------------------------------------------------------------------------
%	DOCUMENT INFORMATION
%----------------------------------------------------------------------------------------

\title{Network Intruder Detection \\
\large APA Practical Work \\
Universitat Politècnica de Barcelona} % Title
\author{Lluc Bové \& Aleix Trasserra} % Author name

\date{Q1 2016-17}

\begin{document}

\maketitle % Insert the title, author and date


%----------------------------------------------------------------------------------------
%	SECTION 1
%----------------------------------------------------------------------------------------

\section{Introduction}
In the are of \textit{computer security}, one of the current works is to protect computer networks from unauthorized users. \\
The goal of this work is to build a system capable of predict if a connection is an \textit{intrussion} ()and which kind of intrussion) or if it is a normal connection.  
\subsection{The data}
Our dataset is a set of \textbf{connection records}.
A \textbf{connection} is a set of TCP packets starting and ending at some well defined times, between which data flows from a \textit{source IP} address to a \textit{target IP} address under some well defined \textit{protocol}.
There are four main attack categories which the system have to identify apart from \textit{normal connections}:
\begin{itemize}
	\item DOS: denial-of-service.
	\item R2L: unauthorized access from a remote machine.
	\item U2R: unauthorized acces lo local superuser(root) privileges.
	\item probing: surveillance and other probing, e.g: port scanning.
\end{itemize}

*The original dataset can be obtained visiting the following link: http://kdd.ics.uci.edu/databases/kddcup99/kddcup99.html
%----------------------------------------------------------------------------------------
%	SECTION 2
%----------------------------------------------------------------------------------------

\section{Related Work}
The paper \textit{Cost-based Modeling and Evaluation for Data Mining With Application to Fraud and Intrusion Detection Results from the JAM Project by Salvatore J. Stolfo, Wei Fun, Wenke Lee, Andreas Prodromids and Philip`K. Chan.} from the Florida Institute of Technology talks about JAM project. \\
The paper describes the results achieved using a JAM distributed data mining system for the problem of fraud detection in financial information systems. \\

Other work related with \textit{network intrussion} problem was reported by Ramesh Agarwal and Mahesh V. Joshi. They described a rule called \textit{PN-rule} which is a framework for learning classifier models and it's study case was network intrussion detection. In the following paper you can read more about this project: PNrule: A New
Framework for Learning
Classifier Models in Data
Mining (A Case-Study in
Network Intrusion
Detection)
by 
Ramesh Agarwal and Mahesh V. Joshi
%----------------------------------------------------------------------------------------
%	SECTION 3
%----------------------------------------------------------------------------------------

\section{Data Exploration Process}
In this section we will take a first look to our data, in order to know what are we going to analyse. First we'll preprocess it, this includes detecting outliers, missing values... And deleting them. Then we are going to perform a feature extraction and selection. This consists in deleting variables that aren't necessary and deriving new ones. Then we'll visualize our data using FDA and finally we'll do a clustering. All this process can be followed using the script provided called \textit{dataExploration.R}.\\

\subsection{Preprocessing}
The original data has two forms. The first one has a dimension of 5 million rows approximately and the other is a reduction of 10\%. We assured that this purge didn't affect the proportion of less used attack types, and it didn't, so from now on we will work with this reduction.\\

First of all when we analyse our dataset we see that variables have no name, so we name them. Then we execute \lstinline|summary| to detect abnormal things on our data, like strange means, medians, extreme values, errors... So we first realize that our categorical variables aren't factors, and they have no text. To solve this first we declare all categorical as factors and we use \lstinline|TRUE| and \lstinline|FALSE| for boolean ones, and we use names for the others. We realize that some variables do not vary so we delete them. We detected an error with a boolean factor, it had missing values so we deleted the rows.


\subsection{Feature extraction/selection}
First we need to define the target variable. We decide to use the main attack in the network. This is a category for each attack type. The variable is defined as table \ref{table:main} shows. 
\begin{table}
	\begin{tabular}{ll}
		Main Attack & Attack Type                                                            \\ \hline
		DOS         & back, land, neptune, smurf, teardrop                                       \\
		U2R         & buffer overflow, loadmodule, perl, rootkit                               \\
		R2L         & ftp\_write, guess password, imap, multihop, phf, spy, warezclient, warezmaster \\
		probe       & ipsweep, nmap, portsweep, satan                                           \\
		normal      & normal                                                                 \\
	\end{tabular}
	\caption{This table shows the main attack classification}
	\label{table:main}
	
\end{table}
Now we have this categorization so we don't need attack type any more but we save it because it may be a useful alternative target respect main attack. 


\subsection{Descriptive Analysis}
We'll proceed to do a descriptive analysis of our data. To make our graphs more beautiful we use \textit{ggplot2}\cite{ggplot}. For every numerical variable we do a histogram and a boxplot. We realize that most variables don't have any.
\subsection{Visualization}
\subsection{Clustering}
In this section we'll perform a clustering analysis in order to see natural groupings of data in our dataset.  \\
We'll use \textit{k-means} algorithm to perform clustering process. Our goal is to see what number of clusters gives us a better explanation of natural groupings in our dataset. \\
To determine which is the best number of clusters, we'll use \textit{Calinski-Harabasz} and  \textit{likelihood} index that gives us information about the quality of a cluster.
With a first look, we've seen that the best number of clusters could be between 40 and 50 clusters. \\
To see it, we'll carry out an experiment whith the following steps:
\begin{enumerate}
	\item With our dataset preprocessed, we'll execute K-means 20 times, one for each value of(K) in range [40..51] and we'll calculate Calinski-Harabasz index for each execution.
	\item Same process for likelihood index.
	\item Take the mean of 20 executions for each value with Calinski-Harabasz index and plot it.
	\item Same process for likelihood index.
	\item Analyse results and extract conlcusions.
\end{enumerate}



The following figures contains the plots that shows the result of the experiment explained before. Figure \ref{calinski} shows the means of the Calinski-Harabasz index for the 20 exectuions of k-means. Figure \ref{likelihood} shows the results for likelihood index. \\

As we can see, Calinski-Harabasz index indicates that better number of clusters is 50 but on the other hand likelihood index indicates that 48 is the best number of clusters. \\
Although we may consider that the best number of clusters is in range between 48 and 50 clusters, if we compare the behavior of the two index, it is clear that it is quite different. \\
We don't know the reason of these different behaviors, but we think that with more executions it could be solved due to random initialization of k-means algorithm leads the index to take unsual values.
\begin{figure}
	\label{calinski}
\includegraphics[scale = 0.75]{../../plots/Calinski}
	\caption{Number of clusters vs Calinski-Harabasz index}
\end{figure}


\begin{figure}
	\label{likelihood}
	\includegraphics[scale = 0.75]{../../plots/likelihood}
	\caption{Number of clusters vs likelihood index}
\end{figure}

%----------------------------------------------------------------------------------------
%	BIBLIOGRAPHY
%----------------------------------------------------------------------------------------

\printbibliography

%----------------------------------------------------------------------------------------


\end{document}